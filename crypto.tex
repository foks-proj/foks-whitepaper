
\section{Cryptographic Design}

Here we describe the important cryptographic decisions at play in the FOKS
system.  For the most part, our bias is toward simplicity and boring,
failure-proof cryptography.  For instance, as decribed in
Section~\ref{sec:merkle}, we use a vanilla collision-resistant hash function to
hide tree locations, rather than the slightly more exotic pseudo-random function
approach. The emphasis throughout is on tried-and-true cryptography that will
prove as robust as possible to misuse through software bugs.

\subsection{The Snowpack Domain-Specific Language}

One of the biggest risks in a system like FOKS is signature malleability
due to issues like permitting non-canonical encodings~\cite{conf/crypto/Bleichenbacher98,bip66}, 
lack of clear domain separation~\cite{ncc2019,cryptoeprint:2020/241}, 
or undefined behavior due to parsing and encoding bugs~\cite{heartbleed2014durumeric}.

To address these threats, and at the same time to provide an convenient language for 
defining RPC protocols, we introduce the Snowpack Language~\cite{snowpack}, which is influenced
by protobufs~\cite{protobuf}, Framed Msgpack-RPC as used in Keybase~\cite{keybase}
and Cap'n Proto~\cite{capnproto}. A further prooperty we insist upon is support
for backwards and forwards compatibility. Since FOKS is a federated system, we have 
no expectations that upgrades will happen in lockstep. The protocol itself must behave
well an any number of partially-upgraded configurations.

\subsubsection{Structures}
\label{sec:snowpack-structures}

By way of example, see Figure~\ref{fig:group-change} for the definition of a sigchain link both for teams
and users in the Snowpack language. Any constant of the form \texttt{@1} or \texttt{0x8fbf37f586b0bc6e}
is meant to be \textit{immutable}. Once written down in the protocol, it should never change. For instance,
look at the first field in the structure: \texttt{chainer @0 : HidingChainer;}. The \texttt{@0} indicates that this
field will take the 0th slot in the encoded version of the structure. Future editors of this file must
never introduce a new field at slot 0 with a different type, as that would cause old clients to fail
in decoding. All new fields should be added at the end of the structure. Old clients will ignore fields
from the future that they do not know how to decode. Similarly, it is allowable to delete a field.
Software with older version of the protocol will get 0 values for the deleted fields. In the Go language,
this means \texttt{0} for integers, empty strings for strings, empty slices for lists, and nil 
pointers for optional fields. Of course new clients must consider the impact on older clients to 
leave 0-ed fields, but the protocol layer itself does not introduce a failure here.

\begin{figure}[ht]
  \centering
\begin{verbatim}
  struct GroupChange @0x8fbf37f586b0bc6e {
    chainer @0 : HidingChainer;
    entity @1 : FQEntity;
    signer @2 : GroupChangeSigner;
    changes @3 : List(MemberRole);
    sharedKeys @5 : List(SharedKey);
    metadata @6 : List(ChangeMetadata);
}
\end{verbatim}
\caption{A sigchain link in the Snowpack language.}
\label{fig:group-change}
\end{figure}

Structures like \texttt{GroupChange} from Figure~\ref{fig:group-change} are
encoded as JSON-style arrays on the wire, with fields written to slots as
directed by their \texttt{@i}-style positions. Elided fields are written down as
\texttt{null} values. Before going out to the wire, the JSON-style arrays are
encoded with the Msgpack~\cite{msgpack} encoding format. Where two possible
encodings are possible (e.g., the number \texttt{0x2} can be encoded as
\texttt{0x2} or \texttt{0xcd 0x00 0x02}), the shorter encoding is mandated. Note
that field names (like \texttt{chainer} above) are not sent over the wire, but
are available on either end as human-readable references to fields. Thus, it is
permitted to rename a field as long as its type doesn't change. We note that
serializing using JSON-style dictionaries seems error-prone, since keys can
repeat or be ordered in different ways. Snowpack's slot-oriented encoding aims
to avoid these styles of ambiguities and to minimize encoding sizes. At the same
time, development tools can decode encoded messages without reference to
protocol specification files.

\subsubsection{Domain Separation}

In the definition of the \texttt{GroupChange} structure from
Figure~\ref{fig:group-change}, note the 64-bit integer
\texttt{0x8fbf37f586b0bc6e}. This is a randomly-generated number that serves as
a \textit{domain separator}. We refer to it below as unique type identifier
(UTID). Though domain specifiers are optional in the Snowpack language, when a
structure provides it, the snowpack compiler fills in five possible
cryptographic operations for the structure:
%
\begin{itemize}

  \item \texttt{PrefixedHash}(\textit{obj}): The object's UTID is big-endian encoded, then
  prepended to the object's binary Msgpack encoding. The hash of the combined message is returned.

  \item \texttt{Hmac}(\textit{obj}, \textit{key}): As above, a mesasge is formed out of the object's UTID
  concatenated with the encoding of the object itself. The combined message is the message input
  to the MAC function, and the key is passed through as the key.

  \item \texttt{SealIntoSecretBox}(\textit{obj}, \textit{nonce}, \textit{key}): The object's UTID is
  encoded and conctacted with the supplied \textit{nonce}. The new value is then used as the nonce
  passed into the encryption algorithm, along with the encoding of \textit{obj} and the supplied \textit{key}.

  \item \texttt{Sign}(\textit{obj}, \textit{key}): The object's UTID is prepended to an encoding
  of \textit{obj}; the combined message is then used as the message input, passed along to the 
  signature algorithm along with the supplied \textit{key}.

\end{itemize}
%
%
Public key encryption calls into \texttt{SealIntoSecretBox} with a random session key, so therefore uses the same
domain separation mechanism. Inverse operations for \texttt{Hmac}, \texttt{SealIntoSecretBox} and \texttt{Sign}
are also provided; they similarly supply UTIDs where necessary to ensure that verification and decryption
succeed.

The programmer must supply their own tooling to generate these UTIDs. Simple CLI tools or 
editor plugins suffice. However, FOKS provides two mechanisms to guararntee they remain unique
across the project. At compile-time, a simple tool examines all input files to guarantee that
no UTID contant appears twice. And at runtime, the compiler provides a list of all UTID constants
compiled from the protocol input files. The program fails an assertion if it sees any repeats.

\subsubsection{Variants}

We have seen Snowpack structures in Section~\ref{sec:snowpack-structures}. Another important
data type is the \textit{variant}, also known as a descriminated union. Figure~\ref{fig:link-inner}
shows an example from FOKS. The enumerated type \texttt{LinkType} has two possible values:
\texttt{GroupChange} for main chains, and \texttt{Generic} for subchains like team membership chains
and user settings chains. Based on the switch value, the enumerated type takes the form of 
a \texttt{GroupChange} or \texttt{GenericLink} object. Variants have many of the same restrictions
as structures: fields specified with `@i'-style slots should never be repurposed, though they 
can be dropped; also, new type possibilites can be added without breaking the protocol.

\begin{figure}[ht]
  \centering
  \begin{verbatim}
    variant LinkInner switch (t : LinkType) @0xacf9066572a9e7de {
      case GroupChange @0 : GroupChange;
      case Generic @1 : GenericLink;
    }\end{verbatim}
  \caption{A variant in the Snowpack language.}
  \label{fig:link-inner}
\end{figure}

\subsection{Cryptographic Primitives}
\label{sec:cryptography}

We have tried as much as possible to make boring, unopinionated cryptographic
decisions.

\subsubsection{Hashing and MAC'ing}

Throughout the system, hashing uses SHA512 truncated to 256
bits~\cite{rfc6234}. Message authentication codes are with HMAC~\cite{rfc2104}
over SHA512/256. HMAC is used for MAC'ing but also for commitments, and in 
general, any context where a pair of items are hashed together (one being the ``key''
and the other being the ``data'').

\subsubsection{Signatures and Post-Quantum Encryption}

For signing, we use EdDSA with the Ed25519 curve~\cite{eddsa}.  Public-key
encryption is a hybrid of Diffie-Hellman over
Curve25519~\cite{cryptoeprint:2011/368} and MLKEM~\cite{nist-fips-203} using a
construction similar to X-Wing~\cite{xwing}, but with a different binary
encoding format and constants. Thus, in practice, all device keys, PUKs,
PTKs, and so are are not a single keypair, but rather a triple: an EdDSA 
keypair, a Curve25519 keypair, and an MLKEM keypair. Wherever public keys
are introduced, an EdDSA signature over the Curve25519 and MLKEM public keys
is produced to bind them together.

The exact derivation of the hybrid encryption secret key is specified in 
Snowpack, using the structure shon in Figure~\ref{fig:xwing}. Hasn inputs are:
the domain separator (UTID); a version number; the shared key exchanged via KEM;
the shared Diffie-Hellman key; the receiver's public keys; and the sender's
public DH key. Though everywhere in the project we use SHA512/256, here we 
use SHA3 to follow the spirit of the X-wing specification.

\begin{figure}[ht]
  \centering
  \begin{verbatim}
struct HybridSecretKeySHA3Payload @0x8a9e327647262289 {
    version @0 : BoxHybridVersion;
    pqKemKey @1 : KemSharedKey;
    dhSharedKey @2 : DHSharedKey; 
    rcvr @3 : HEPK; // Hybrid Encryption Public Key = DH + KEM Public keys
    sndr @4 : DHPublicKey;
}
  \end{verbatim}
  \caption{Hybrid encryption secret key derivation in the Snowpack language.}
  \label{fig:xwing}
\end{figure}

\subsubsection{Key Derivation}
\label{sec:xwing-key-derivation}

As described just above, each public key in FOKS actually consists of three keypairs. However,
a single 32-byte secret seed suffices to generate all three, which simplifies secret key
management and backup keys.  The key derivation system again uses the Snowpack specification
system and simple HMAC-based key derivation. Figure~\ref{fig:key-derivation} shows the 
Snowpack structures and variants used. The derived key is the HMAC of the \texttt{KeyDerivation} object
with the secret 32-byte seed as the key.

\begin{figure}[ht]
  \centering
  \begin{verbatim}
enum KeyDerivationType {
    // Core types
    Signing @0;
    DH @1;
    SecretBoxKey @2;
    MLKEM @4;
    AppKey @5; // Used for different higher-level applications, like KV Store
}

enum AppKeyDerivationType {
    Enum @0;
    String @1;
}

enum AppKeyEnum {
    KVStore @0;
}

variant KeyDerivation switch (t: KeyDerivationType) @0xd35cdcc95caef674 {
    case MLKEM @4: Uint; // need 2 32-byte values to get a 64-byte seed
    default: void;
}
\end{verbatim}
  \caption{Structures and variants used in key derivation.}
  \label{fig:key-derivation}
\end{figure}

For example, to make a new PTK, the team adminstrator picks a random 32-byte
seed value.  When ever the PTK is used in a symmetric context (like for
encrypting older PTKs), the key derivation uses \texttt{KeyDerivationType =
SecretBoxKey} as an HMAC input.  Similarly for using the PTK as a signing key.
All derived keys are also 32-bytes with the exception of the ML-KEM key, which
needs 64-bytes. The two halfs of this derived key are generated with the same
mechanism, but using \texttt{MLKEM=0} and \texttt{MLKEM=1} in the
\texttt{KeyDervitation} object as the HMAC input.

\subsubsection{\Yubis}

FOKS supports hardware keys like YubiKey that support the Personal Identity
Verification~\cite{nist-sp-800-73-5} (PIV) standard.  This standard allows the
device to perform public key cryptopgrahic operations, like ECDSA and
Diffie-Hellman over the p256 elliptic curve~\cite{nist-fips-186-3}. Though we
have chosen the Ed25519 and Curve25519 curves for use everywhere, we now need to
accommodate another curve to fit the PIV standard. Too much ``agility'' has
proven problematic for other systems~\cite{jwt-none}, and we would like to avoid
it as much as possible with FOKS, but we make an exception here for a popular
hardware standard.

The bigger issue with YubiKeys is: what do to about post-quantum security?  To
date, we have not seen a wide release of an algorithm like ML-KEM to hardware
devices, and even if so, we'd like to support older, widely-deployed hardware. 

Since there are no perfoct solutions here, we have designed a PQ-secure system
around existing PIVs as follows:

\begin{enumerate}
  \item Extract a "secret" from the PIV module: pick an unused ``retired key management'' slot
  (0x82-0x95), and compute $g^{x^2}$ via the ECDH algorithm. Use this value as the seed
  to create a new ML-KEM keypair. Compute ML-KEM on user's computer after extracting the secret
  and deriving the keys.
  \label{step:pq-secret}

  \item Select a different retired key management slot to use for classical ECDH over curve p256.
  Compute ECDH as usually using the YubiKey's hardware.

  \item  Combine the secret keys from the previous two steps using the X-wing-style
  derivation scheme from Section~\ref{sec:xwing-key-derivation}.
\end{enumerate}

An important property of this system is that all of the relevant key material
lives on the YubiKey; none lives on the user's computer. The YubiKey is all the
user needs to recover important secrets, even if the computer is lost or suffers
data loss. Further, this system is no less secure than an encryption scheme
without PQ-security, as the classical ECDH computation still happens on the
YubiKey as normal. That is, we are not forcing the user to choose between PQ and
hardware security.  However, this scheme has an important shortcoming. If the
user later reuses the key management slot in Step~\ref{step:pq-secret} for a
different purpose, and exports the public key $g^x$ from the device, the scheme
is no longer PQ-secure.  A quantum computer could recover $x$ from $g^x$ and
then derive the ML-KEM secret key.  This shortcoming makes us long for better
hardware support for ML-KEM.  However, a mitigating factor here is that PIV is
an infrequently-used standard (and for instance is way less popular than FIDO2).
There are few competing applications using these features and key slots.

\subsubsection{High-Entropy Secret Phrase}
\label{sec:hesp}

For backup device keys (which FOKS users can write on pieces of paper), and 
exchanging provisioning secrets betwewen two computers, FOKS uses a simple
encoding scheme called the ``high-entropy secret phrase''. The pattern is 
a series of random words, each separated by a random number. All
words are chose from the BIP39 wordlist~\cite{bip39}. The exact
parameters depend on the application, and are shown in Table~\ref{tab:hesp-params}.

\begin{table}[ht]
  \centering
  \begin{tabular}{|c|c|c|c|c|c|}
    \hline Application & \# of Words & \# of Numbers & Number Range & Entropy \\
    \hline
    \hline
    Provisioning & 7 & 6 & $[0,2^8-1]$ & $7\cdot 11 + 6\cdot 8 = 125$ \\
    \hline
    Backup Device Key & 8 & 7 & $[0,2^{13}-1]$ & $8\cdot 11 + 7\cdot 13 = 179$ \\
    \hline
  \end{tabular}
  \caption{Parameters for high-entropy secret phrases.}
  \label{tab:hesp-params}
\end{table}


\subsection{Secret Key Management}

Secret keys derives from 32-byte seeds, which never leave the device they are created
on (with the possible exception of backup keys, which are written down on paper).
We discuss here how the FOKS client stores these secret seeds persistently. 

\subsubsection{Secure Enclaves}

Where possible, FOKS uses OS-specific secure enclaves. This is the simple case.
FOKS stores the actual 32-byte seeds in a FOKS-specific keyring file in the user's 
home directory. For each seed, FOKS picks a random 32-byte key to encrypt with, and,
if possible, stores that 32-byte key in the user's OS keyring. 

\subsubsection{Passphrase-based Management}

Though it's not encouragd, FOKS does offer a passphrased-based protection mechanism
for secret key seeds. As above, each secret seed gets its own secret-key wrapping 
material wrapping key (SKMWK). But instead of storing the SKMWKs in the OS keystore,
they are encrypted with a key derived from the user's passphrase. We have important 
design considerations for this system that make it quite complex:

\begin{enumerate}

\item If the user has two computers, $A$ and $B$, and the user changes his passphrase on A, when B comes online
 with the old passphrase, it has to decrypt with the new passphrase.

\item Keys encryped for old passphrases need to eventually be migrated to the new passphrase, so that if
an attacker gets the old passphrase and all server data, they still can't decrypt the key.  Of course
this is only possible if that computer $B$ comes back online after the change, but assuming that
the property should hold.

\item Passphrae recovery: to change the passphrase and recover keys, it is sufficient to know the latest PUK.
 Thus, having a backup paper key or a backup YubiKey should suffice to ``recover'' a passphrase
 and to allow the user to change it without knowing the old passphrase.

\item As with passphrases, if the PUK is updated, all machines with passphrase-encrypted keys
  should eventually rotate (when they come online) so that they cannot be decrypted with an old PUK
\end{enumerate}

\label{sec:passphrase}

\newcommand{\secretBox}[2]{\textsf{secretBox}(#1, #2)}
\newcommand{\dhbox}[3]{\textsf{dhBox}(#1, #2, #3)}
\newcommand{\pk}[1]{\textsf{publicKey}(#1)}
\newcommand{\sk}[1]{\textsf{secretKey}(#1)}

We describe the process through a small example: two rotations, one due to a PUK rotation,
and one due to a passphrase change. The net result is three different configurations
(the original, and the two following rotations). The general idea is that we have a new ``session''
key at every update, which is symmetrically encrypts the SKMWKs. The session key gets encrypted
twice: once for a key derived from the current passphrase, and one for the user's PUK. This
allows recovery of the SKMWKs with either the passphrase or the PUK:
%
  \begin{center}
  \begin{tabular}{|c|c|c|c|}
    \hline
     Key & Epoch 0 & Epoch 1 & Epoch 2 \\
     \hline
     \hline
      SKMWK & $r_0$ & $r_1$ & $r_2$ \\
      Session Key & $s_0$ & $s_1$ & $s_2$ \\
      Ephemeral DH Key & $t_0$ & $t_1$ & $t_2$ \\
      Passphrase & $p_0$ & $p_0$ & $p_1$ \\
      PUK & $u_0$ & $u_1$ & $u_1$ \\
     \hline
  \end{tabular}
\end{center}
%
At Epoch 0, we have the initial configuration, which consists of the following three encryptions:
%
\begin{align*}
e_0 &= \secretBox{r_0}{s_0} \\
f_0 &= \dhbox{s_0}{\pk{p_0}}{\sk{t_0}}, \pk{t_0} \\
g_0 &= \secretBox{[s_0,\pk{p_0}]}{u_0} \\
\end{align*}
%
$e_0$ is the encryption of the SKMWK $r_0$ for the session key $s_0$.  $f_0$ is the encryption 
of the session key for the user's current passphrase, $p_0$. To derive $\sk{p_0}$, we employ
a simple stretching algorithm and interpret the result as a Curve25519 secret key; then we
derive $\pk{p_0}$ from the secret key as usual. Finally, $g_0$ is the encryption of the session
key $s_0$ for the user's current PUK. We include $\pk{p_0}$ in the plaintext for reasons
we will see shortly.
%

