\documentclass[11pt]{article}

\usepackage[utf8]{inputenc}
\usepackage{graphicx}
\usepackage{amsmath}
\usepackage{hyperref}
\usepackage{listings}
\usepackage{geometry}
\usepackage{xcolor}

\geometry{
    a4paper,
    margin=1in
}

\title{The Federated Open Key Service (FOKS)}
\author{Maxwell Krohn (max@ne43.com)}
\date{\today}

\begin{document}

\maketitle

\begin{abstract}
This paper presents FOKS (Federated Open Key System), a decentralized key management system designed to provide secure and flexible key distribution across federated networks. We describe the system architecture, security model, and implementation details.
\end{abstract}

\section{Introduction}
% Introduction content goes here

\section{Background}
% Background content goes here

\section{System Architecture}
% Architecture content goes here

\section{Security Model}
% Security model content goes here

\section{Implementation}
% Implementation details go here

\section{Evaluation}
% Evaluation content goes here

\section{Related Work}

The initial inspiration for FOKS is the SUNDR project~\cite{sundr}, which first
originated the idea of a fork-consistent blockchain of edits facilitated by a
untrusted server.  Like Keybase~\cite{keybase}, FOKS applies this basic
architecture to the problem of key distribution, rather than the data those keys
might secure. Many other projects have riffed on this, from
CONIKS~\cite{coniks}, to the SEAMless work out of Microsoft Research, to the
widespread adoption of Key Transparency Signal, WhatsApp and iMessage.
The question of federation has largely been ignored, as these sytems all
shared the basic architecture of a single upstream server.

\section{Conclusion}
% Conclusion goes here

\bibliographystyle{plain}
\bibliography{refs}

\end{document}
