\documentclass[11pt]{article}

\usepackage[utf8]{inputenc}
\usepackage{graphicx}
\usepackage{amsmath}
\usepackage{hyperref}
\usepackage{listings}
\usepackage{geometry}
\usepackage{xcolor}
\usepackage{enumitem}
\usepackage{amsfonts}
\usepackage{tikz}
\usetikzlibrary{calc}

\geometry{
    letterpaper,
    margin=1in
}


\title{The Federated Open Key Service (FOKS)}
\author{Maxwell Krohn (max@ne43.com)}
\date{\today}

\begin{document}

\newcommand{\yubi}{Yubikey}
\newcommand{\Yubi}{Yubikey}
\newcommand{\yubis}{Yubikeys}
\newcommand{\Yubis}{Yubikeys}

\maketitle

\begin{abstract}

This paper presents FOKS (Federated Open Key System), a
decentralized key management system designed to provide secure and flexible key
distribution across federated networks. The basic problem addressed is that of
of two parties sharing end-to-end encrypted data across the internet, where both
parties have several devices. They might rotate devices, form mutable teams with
other users, or even teams of teams in an arbitrary graph. They need to share
secret key material to facilitate symmetric encryption, and this material must
rotate whenever devices are replaced, or team membership changes.  This is a
very natural problem but one that still lacks an adequate solution.  Morever, we
believe key management should not lock users into a particular, walled provider,
but instead, should allow for federation and independent management of server
resouces, as we see in HTTP and SMTP.  We describe the system architecture,
security model, and implementation details of a system that achieves secure,
federated key exchange, and enables useful applications like end-to-end
encrypted data sharing and git hosting.

\end{abstract}


\section{Introduction}

In recent years, Signal, iMessage and WhatsApp have proven out the customer
demand for end-to-end encryptied communication. But despite the success of these
systems, and their vast improvements over previous, less-secure systems,
important problems remain unresolved.  Most obviously, questions remain around
identity. In a recent, high-profile incident, United States government officials
misused Signal to mistakenly leak matters of national security to the
press~\cite{signal-hesgeth-leak}. Relying on phone numbers as identifiers is
only part of the problem; the larger issue, arguably, is that public identities
are hard to audit and map to public keys, and that large groups are even harder
to get right.

We see other issues: all the systems mentioned above and many others lock users
into a walled-garden with a single provider. Such a closed-system, and often
closed-source ethos runs counter to the early philosophy that drove the original
PGP cyberpunk movement: fear of a central authority.  There are also practical
considerations, that closed-source systems are more difficult to audit. And that
single-provider systems suffer from vendor lock-in and have the leverage
to``monetize'' their users, since switching to a competing platform is
impossible.

The authors have deep experience with the Keybase system~\cite{keybase}, which
came at these problems from a different angle. In Keybase, the initial focus was
on identity, multi-device support, and formation of auditable groups that could
evolve over time. But Keybase of course has limitations too, and like Signal,
WhatsApp and iMessage, is stuck on a single-provider model.

This paper introduces a new system: the Federated Open Key Service (FOKS).  It
inherets much from these prior system but inhabits a different point in the
design space. FOKS provides secure key distribution for users who have multiple
devices. It allows those users to form groups, and unlike previous systems, for
those groups to join other groups. Whether managing a user's device cloud, or
managing the membership of a team, FOKS ensures that a malicious server cannot
inject invalid members, and that it cannot withhold import revocations and
deletions.  Many of these features are possible with previous systems, but
FOKS's major advance architectural advance is the support of federation. Anyone
can run a server in the FOKS network. Users can stay siloed on different
servers, or can form teams that span multiple servers. Federation gives users
more choice, control and better guarantees. Though servers cannot decrypt or
sign on behalf of users, they still can see metadata, and often are called upon
to protect user privacy. Therefore, companies or small, tight-knit groups have
good reason to run their own servers, and can do so in the system.

This paper introduces and describes the FOKS system. We cover a threat model
in Section~\ref{sec:threatmodel}, a system design in Section~\ref{sec:design}, 
the use of cryptography in~\ref{sec:crypto}, and some important applications
in Section~\ref{sec:apps}. The primary goal here is not academic novelty, but
rather to describe a system that occupies a unique
and quite useful set of trade-offs for end-to-end encrypted systems.
However, there are some, to our knowledge, new contributions:
%
(1) An exploration of key-rotation for teams that can form
      nearly arbitrary graphs across federated servers;
%
(2) a system for hiding identity and team updates in a larger
      transparency tree with basic crypto primitives and without the need
      for pseudo-random functions; and
%
(3) a new protocol specification language (called Snowpack) that enforces
  domain separation for cryptographic operations.



\section{Threat Model}

In FOKS, we consider a threat model similar to that of the Keybase~\cite{keybase},
SEAMLess~\cite{chase2019seemless} or CONIKs~\cite{melara2015coniks} systems.
The high level north star is end-to-end secrecy and integrity. Only the clients
at the edges of the system should be able to decrypt important data, and only
those clients can make authorized changes to the data. Of course, multiple
devices per user and mutable groups complicate the picture.

We assume that clients are trustworthy, and behave properly. If this assumption
is violated, say, if a client is compromised by a rootkit, then we cannot
offer any guarantees. 

Users might sometimes lose their devices. In an ideal world, hardware protections
would prevent whoever recovered the device from accessing the device's private
key material. In the case of hardware keys (like YubiKeys), or backup-keys
written on paper, the user has less protection during comrpomise. Regardless,
once the user revokes the lost device, keys should rotate so that data is secure
going forward (this property is known post-compromise security). In some cases,
past data might be safe from the attacker (this property is known as forward-secrecy).
but the specifics depend on the trustworthiness of the server (see below). Similarly, revoked
keys on lost devices lose their signing power, and other devices will not accept
their signatures going forward.

The threat model is here is similar but not exactly the same as Signal's and
WhatsApp's, because our applications feature persistent (rather than ephemeral)
data. If a new user joins an existing group, or if a user adds a new device,
they should be able to access old data, which might be required to reassemble the
shared resource. For instance, when Alice adds Bob to a git project, Bob
should see all past commits in the commit history, otherwise the
application will break. Thus, we can't guarantee foward-secrecy, since
lack of forward secrecy is needed for the application to function properly.

In FOKS, clients pick their servers. They might select for servers
that are generally aligned with. They can run their own servers, or pick 
from third party hosting providers. Users should assume that servers
are generally trustworthy, but might suffer comrpomises from time to time.
For instance, servers might be running on cloud infrastructure, and the underlying
storage, network, or computation might be compromised. Insiders or state actors
might have privileged access to the underlying infrascture. 

If servers behave honestly, the FOKS system works securely as expected.  If
servers behave maliciously, they can deny access to data through a variety of
mechanisms: they can go offline, they can withoold data, or they can subtly
current server-resident data to confuse clients. In this last case, the system's
security design should prevent the clients from leaking secrets or accepting
unauthorized changes to data. But as in the other more obvious cases, the
clients will lose access to their data.

When servers are behaving honestly, they can provide clients with
forward-secrecy. That is, if honest servers throw away data encrypted with old
keys, an attackets with access to private keys cannot rcannot ecover past data.
This property is an improvement over that offered by the Keybase system, which
assumed the worst in the case of a comrpomise. If we assume on the other hand
that an attacker who steals a private key operates in cohoots with the server,
then we cannot offer any guarantees about forward secrecy.

Servers do not trust each other. If one server becomes corrupted, it has no
bearing on the other servers in the system. In other words, we assume attackers
can stand up their own servers, since anyone in the system can do so.


% Introduction content goes here


\section{Design}

FOKS is a classic client-server system. At a hight level, the clients
manage private keys, and the server manages public keys, encryptions of, 
shared secret keys, and encrypted data. Users generally trust their
servers to be online, available and not to intentionally sabotage
agreed-upon protocols. 

\subsection{System Architecture}

Much like HTTP or STMP, FOKS clients communicate with one or more servers,
depending on where users have accounts. They can safely ignore the other servers
in the system. Most communication is between client and server, and there is
little if any server-to-server or client-to-client communication.  This
property simplifies protocol upgrades and network configuration.

Each client can speak for many users, as users can have accounts on different
servers, or several accounts on the same server. By analogy, an email client can
server multiple emails accounts for the same user concurrently, say one for work
and one for personal use. Or a web browser might have different personae (with
different cookies, preeferences, passwords and history) for the same user.

Each of the users can of course have multiple devices, like a desktop, a laptop,
a phone, and a YubiKey. Additionally, users can have "backup devices", which can
be writted down on paper and stored in a safe place. The system recommends at
least two devices to prevent data loss. That is, these devices have private keys
that decrypt data, and the loss of the last key prevents decryption of the data.
Obviously there is a trade-off here: the more devices, the more likely the user
will lose one, or have one stolen; the fewer devices, the more likely the user
will lose all devices and therefore access to data. Some optimal middle ground
exists, but varies with the users and their behaviors. 

\subsection{Key Hierarchy}

The FOKS key hierarchy sits at the core of the system. It aims to provide users
with a sequence of symmetric keys shared across all of their devices, so that
they can store data encrypted with the latest key, and can decrypt (and
authenticate) data encrypted with older keys when necessary. Similarly, users in
a team should share secret keys that users outside their teams cannot see, allowing
them to share encrypted data via untrusted FOKS servers.

\subsubsection{Device Keys}

When a user sits down at a FOKS client to signup or provision a new device
for an existing account, she first creates a new key-pair specifically
for that device. The private key never leaves the device. She shares the public key
with the FOKS server, who eventually selectively shares it with user users.
We detail the exact cryptography in Section~\ref{sec:cryptography}.

Hardware keys that support the PIV protocol (like YubiKey version 5 and later)
can also be used as device keys. These devices get randomly-generated private
keys in the factory, written to one of 20 possible "slots." FOKS users select
a slot to use, and the client sends the corresponding public key to the FOKS 
server. Signing and decryption operations happen on the device against the chose
slot. 

\subsubsection{Per-User Keys (PUKS)}

Every user on the FOKS system has one of more per-user keys, or PUKS. A PUKS
is a randomly-generated key-pair whose private key is encrypted for each of
the device public keys. This way, all current devices can access the current
PUK secret key, and perform decryptions or signatures for the current PUK public
key. The client makes a new PUK every time the user revokes a device. The system
encrypts the old PUK secret keys for the new PUK secret key. This way,
a device that has access to the latest PUK can get access easily to all prior
PUKs.

Once the PUK sequence is estblished, the system has a convenient way to encrypt
a data for all of the user's device --- it simply encrypts the data for the 
user's latest PUK.

\subsubsection{Per-Team Keys (PTKs)}

Each team has a sequence of per-team-keys, or PTKs, which are analogous
to PUKs for users. Upon creation, a team gets a new random PTK. The 
client performing the creation sends the public part of the PTK to the server.
The private part of the PTK is encrypted for each member's latest PUK,
and therefore is available on each of the user's devices.

As with PUKs, data that the team shares is encrypted for the team's latest PTK,
and all members can decrypt it. As we will see in Section~\ref{sec:teams}, teams can
join other teams, but the key hierarchy works just the same. When team $A$ joins
team $B$, the secret part of team $B$'s PTK is encrypted for team $A$'s latest PTK,
so that all members of team $A$ can decrypt $B$'s PTK, and therefore, all of
$B$'s encrypted data.

\subsection{Key Roles}

\newcommand{\owner}{\textsf{owner}}
\newcommand{\admin}{\textsf{admin}}
\newcommand{\reader}{\textsf{reader}}
\newcommand{\none}{\textsf{none}}
\newcommand{\role}[1]{\textsf{role}(#1)}


FOKS has a notion of a ``role'' for device keys, PUKs and PTKs. The roles
are: \owner, \admin, and \reader, but \reader{} keys have a "visibility level"
that varies between -32768 and 32767. There is a total ordering among key roles,
so that $\owner > \admin > \reader$, and between reader keys, $k_1 > k_2$ iff
$k_1$ has a higher visibility level than $k_2$.

The important property enforced is that we only encrypt PUK $k$ for device key
$j$ if $\role{k} \le \role{j}$, and similarly, we only encrypt PTK $k$ for PUK
$j$ if $\role{k} \le \role{j}$.

The idea here is that the owners of a group get to see all the keys; the admins
can see the admin and reader keys; and the readers can see keys at or below
their visiblility level. This configuration allows groups to have
lower-privileged members, and for users to have lower-privileged devices. At
Keybase, a similar but less-flexible property allows ``bots'' into teams, so that
all the members of the teams can interact with the bots, but the members had
channels to communicate that the bots aren't privvy to.  For now, all user
devices are at the \owner{} role, but we plan to relax this requirement in the
future.

\subsection{Data Structures}

We now have some basic motivation as to what the key system ought to achieve.
It ought to allow groups of devices, groups of users, or groups of users
and teams to share a secret encryption key. From there, they can share data
encrypted (and authenticated) with that key. But the question becomes,
how are users formulated from devices, and how are teams formulated from users
so that only desired members are in the group, especially if the server
behave maliciously?

For instance, a malicious server might fool a user into encrypting secret data
for an invalid device, or team administrator into encrypted data for an invalid
user.

\subsubsection{Signature Chains}

FOKS uses the same mechanism as Keybase here --- the signature chain (or sigchain for short).
The sigchain is a series of signed statements that form a cryptographic chain, meaning they
can only be replayed in the intended order. Replaying the chain allows a viewer to
confirm the chain appears how the author intended and wasn't tampered with, even if
the set of signers varies over time. Of course, signers do vary over time as
users add and remove devices, or as they add and remove members from teams.

Each user (and team) gets its own sigchain. The sigchain keeps an indellable record
of which keys can update the chain, and which PUKs or PTKs are currently
active for the user (or team).

\paragraph{Users} The first link in a sigchain is called the ``eldest'' link. For user sigchains,
the first device generates this link, generates the first PUK, and then computes
a signature over the following data:

\begin{enumerate}\itemsep0em
    \item \label{item:prev} The hash of the previous link in the chain (nil for the eldest)
    \item The current sequence number of the sigchain (which is 1 for the eldest link)
    \item A random commitment to the next tree location (see Section~\ref{sec:location-hiding})
    \item The current Merkle root hash (see Section~\ref{sec:merkle})
    \item The user's ID and the server's host ID (see Section~\ref{sec:hostchains})
    \item \label{item:puk} The user's new PUK public keys
    \item The user's new device key
    \item A ``subchain tree location seed commitment`` (see Section~\ref{sec:subchains})
    \item A cryptographic commitment to the user's username (see Section~\ref{sec:commitments})
    \item A cryptographic commitment to the user's device name (picked by the user)
    \item The role of the new device (currently always \owner ).
    \item For Yubikeys, a public ``subkey' (see Section~\ref{sec:authentication})
\end{enumerate}

The client computes nested signatures first by the new PUKs introduced in
Step~\ref{item:puk}, and lastly by the user's device key. (Recall that sometimes
several PUKs can be introduced at once due to the different possible device
roles). The client uploads the whole package as the user's eldest link.

Subsequent links proceed in largely the same way, with a few minor differences.
The previous hash (\ref{item:prev}) is the collision-resistent hash of the 
package uploaded in the previous step. In some cases, like device addition,
new PUK public keys (\ref{item:puk}) do not appear. In these cases,
no signatures with PUKs are required.

For any link in the chain, a set of devices is authorized to make further
updates to the chain. After the first link, the set contains only the first
device (sometimes called the ``eldest`` device). A link can either add a new
device, or revoke an existing device, updating the set of authorized devices
accordingly. When clients upload new chainlinks, the server enforces valid
signatures by authorized devices. When users replay this chain, they perform the
same check. This simple mechanism ensures the server can't introduce a bogus
device.

\paragraph{Teams} A team chain link contains the following fields, many of which
are analagous to user chains:

\begin{enumerate}\itemsep0em
    \item The hash of the previous link in the chain (nil for the eldest)
    \item The current sequence number of the sigchain (starting at 1)
    \item A random commitment to the next tree location
    \item The current Merkle root hash 
    \item The team's ID and the server's host ID
    \item The user (or team) ID, host ID, and PUK (or PTK) of the actor making the change
       \label{item:team-actor}
    \item New PTK public keys
    \item \label{item:membership} A set of membership changes
    \item A ``subchain tree location seed commitment``
    \item A cryptographic commitment to the team's name (optional if not changing)
    \item The team's ``index range'' (see~\ref{sec:team-index-range})
\end{enumerate}

Since teams can contain both users and other teams, the actor creating or
modifying the team can be either a user or a team. In FOKS, a \textit{party}
refefes to someone or something that can be in a team, so either a user or a
team. In field~\ref{item:team-actor}, the link contains the unique identifier of
the party (which is the user or team ID plus the host ID), and also the key
making the change. For users, this key is the user's latest PUK at the \owner{}
role.  For teams, it's the team's latest PTK at the desired source role. That
is, consider a teams $T$  where users $a$ and $b$ are owners of $T$, $c$ is an
admin and $d$ is a reader (at visibility level 0). If $T$ creates a new team $U$
with source role of \owner, then only users $a$ and $b$ will have access. If
$T$ creates the new team with source role of \reader, then all users will have
access. 

FOKS clients and servers enforce these access controls with the key hierachy.
In the case of the owners of $T$ creating $U$, $T$'s \owner{} PTK appears in
field~\ref{item:team-actor} and performs the signature over the chainlink. 
As $T$ creates $U$, it makes new PTKs for $U$. It encrypts the secret keys
of these new PTKs for the \owner{} PTK of $T$. This way, everyone in the owner
group of $T$ can now access $U$'s PTKs. The second example follows similarly,
with the readers of $T$ getting access to $U$'s secret PTKs after team creation.

The membership changes field (\ref{item:membership}) contains the following
fields for each member being modified:

\begin{enumerate}\itemsep0em
    \item The ``destination role'': the role the member is to have in the team; for removals,
      this is the role \none .
      \label{item:dst-role}
    \item The member's party and host IDs (a party ID is a user ID or a team ID).
      \label{item:party-id}
    \item The source role: the role the member has in its current party.
    \item The member's public PTK or PUK from its current party.
      \label{item:team-member-ppk}
    \item The generation number of that PTK or PUK.
    \item A commitment to a ``team removal key'' (see Section~\ref{sec:removal-keys})
\end{enumerate}

In the case of team creation, member addition, member role upgrade or downgrade, the
role in field~\ref{item:dst-role} is the new role in the target team that the member
has after the change is applied.  In the case of removal, the role is \none. 

Note that in field~\ref{item:team-member-ppk}, the public PUK (or PTK) appears directly
in the chain, in addition to the ID of that party. Admins and owners are later allowed to
make team modifications, and these are the public keys that will sign these modifications.
Team readers in particular might lack the permission to load the chains of these users
and teams directly, so it's crucial the keys appear directly in the team chain. 
See Section~\ref{sec:sigchain-viz} for further details about sigchain visibility.

As alluded to above, owners have ultimate control over the team. They can add and remove
members, add other owners, downgrade owners to admins, etc. Admins have more limited 
control; the have similar control over admins and readers, but cannot upgrade admins
to owners, introduce new members as owners, remove existing owners or downgrade 
existing owners. Readers cannot make any team modifications but can of course read
team chains, and can access data protected by the reader's PTKs at their level
and below.

Teams, unlike users, can include members located on different servers. Above, in 
item~\ref{item:party-id}, we include the host ID of the party in the membership change.
Remote members cannot be admins or owners, but can be readers. This configuration allows
for convenient data sharing across federation boundaries, but simplifies team management
relative to an alternative system where remote members can be owners or admins.

Parties making changes to team sign new team chain links much the same way as users
sign user chain links. First, all new PTKs sign the chain link, and then the acting
party's latest PUK or PTK signs the chain link. This PUK or PTK must be the exact key
advertised earlier in the chain in the case of link 2 and above. The eldest link
is essentially self-signed.

\subsubsection{Hostchains}
\label{sec:hostchains}

Servers maintain hostchains so they can manage and rotate their signig keys, DNS names,
and TLS keys. Like team and user chains, hostchains form a cryptographic chain, ensuring
they can only be replayed in the intended order, even if modified in transit. Chain links
have sequence numbers and contain the cryptographic hashes of previous links. When an
administrator creates a new server, they first create a \textit{hostkey}, a signing key-pair.
This public key becomes the host's ID. The first chainlin contains this hostkey and several
subkeys, one that serves as a TLS CA for the server, and one used to sign \textit{zonefiles}
for the server. The \textit{zonefile} contains the DNS names for the server's various services
(see Section~\ref{sec:foks-server}). Subsequent chainlinks can change any of these keys
or subkeys, as long as they are signed with keys valid up until that point. clients
play these links back to map host IDs to DNS names as they establish connections
to new servers. 

\subsubsection{Merkle Tree}
\label{sec:merkle}

\subsubsection{Location Hiding}
\label{sec:location-hiding}

\subsubsection{Subchains}
\label{sec:subchains}

\subsubsection{Naming}

\subsection{Provisioning}

\subsubsection{Device-to-Device}

\subsubsection{Yubikey-to-Device}

\subsubsection{Device-to-Yubikey}

\subsection{Teams}
\label{sec:teams}

\subsubsection{Cross-Server Teams}

\subsubsection{Invitation Sequence}

\subsubsection{Team Index Ranges}
\label{sec:team-index-range}

\subsection{Key Rotations}

\subsubsection{Removal Keys}
\label{sec:removal-keys}

\subsection{Cryptographic Primitives}
\label{sec:cryptography}

\subsubsection{Key Derivation}

\subsubsection{Yubikeys}

\subsubsection{PQ-KEM and PIV Support}

\subsection{Privacy}

\subsubsection{Blinding and Commitments}
\label{sec:commitments}

\subsubsection{Sigchain Visibility and Permissions}
\label{sec:sigchain-viz}

\subsection{Secret Key Management}

\subsubsection{Secure Enclaves}

\subsubsection{Passphrase-based Management}

\subsection{Beacon Server}


\section{Cryptographic Design}

Here we describe the important cryptographic decisions at play in the FOKS
system.  For the most part, our bias is toward simplicity and boring,
failure-proof cryptography.  For instance, as decribed in
Section~\ref{sec:merkle}, we use a vanilla collision-resistant hash function to
hide tree locations, rather than the slightly more exotic pseudo-random function
approach. The emphasis throughout is on tried-and-true cryptography that will
prove as robust as possible to misuse through software bugs.

\subsection{The Snowpack Domain-Specific Language}

One of the biggest risks in a system like FOKS is signature malleability
due to issues like permitting non-canonical encodings~\cite{conf/crypto/Bleichenbacher98,bip66}, 
lack of clear domain separation~\cite{ncc2019,cryptoeprint:2020/241}, 
or undefined behavior due to parsing and encoding bugs~\cite{heartbleed2014durumeric}.

To address these threats, and at the same time to provide an convenient language for 
defining RPC protocols, we introduce the Snowpack Language~\cite{snowpack}, which is influenced
by protobufs~\cite{protobuf}, Framed Msgpack-RPC as used in Keybase~\cite{keybase}
and Cap'n Proto~\cite{capnproto}. A further prooperty we insist upon is support
for backwards and forwards compatibility. Since FOKS is a federated system, we have 
no expectations that upgrades will happen in lockstep. The protocol itself must behave
well an any number of partially-upgraded configurations.

\subsubsection{Structures}
\label{sec:snowpack-structures}

By way of example, see Figure~\ref{fig:group-change} for the definition of a sigchain link both for teams
and users in the Snowpack language. Any constant of the form \texttt{@1} or \texttt{0x8fbf37f586b0bc6e}
is meant to be \textit{immutable}. Once written down in the protocol, it should never change. For instance,
look at the first field in the structure: \texttt{chainer @0 : HidingChainer;}. The \texttt{@0} indicates that this
field will take the 0th slot in the encoded version of the structure. Future editors of this file must
never introduce a new field at slot 0 with a different type, as that would cause old clients to fail
in decoding. All new fields should be added at the end of the structure. Old clients will ignore fields
from the future that they do not know how to decode. Similarly, it is allowable to delete a field.
Software with older version of the protocol will get 0 values for the deleted fields. In the Go language,
this means \texttt{0} for integers, empty strings for strings, empty slices for lists, and nil 
pointers for optional fields. Of course new clients must consider the impact on older clients to 
leave 0-ed fields, but the protocol layer itself does not introduce a failure here.

\begin{figure}[ht]
  \centering
\begin{verbatim}
  struct GroupChange @0x8fbf37f586b0bc6e {
    chainer @0 : HidingChainer;
    entity @1 : FQEntity;
    signer @2 : GroupChangeSigner;
    changes @3 : List(MemberRole);
    sharedKeys @5 : List(SharedKey);
    metadata @6 : List(ChangeMetadata);
}
\end{verbatim}
\caption{A sigchain link in the Snowpack language.}
\label{fig:group-change}
\end{figure}

Structures like \texttt{GroupChange} from Figure~\ref{fig:group-change} are
encoded as JSON-style arrays on the wire, with fields written to slots as
directed by their \texttt{@i}-style positions. Elided fields are written down as
\texttt{null} values. Before going out to the wire, the JSON-style arrays are
encoded with the Msgpack~\cite{msgpack} encoding format. Where two possible
encodings are possible (e.g., the number \texttt{0x2} can be encoded as
\texttt{0x2} or \texttt{0xcd 0x00 0x02}), the shorter encoding is mandated. Note
that field names (like \texttt{chainer} above) are not sent over the wire, but
are available on either end as human-readable references to fields. Thus, it is
permitted to rename a field as long as its type doesn't change. We note that
serializing using JSON-style dictionaries seems error-prone, since keys can
repeat or be ordered in different ways. Snowpack's slot-oriented encoding aims
to avoid these styles of ambiguities and to minimize encoding sizes. At the same
time, development tools can decode encoded messages without reference to
protocol specification files.

\subsubsection{Domain Separation}

In the definition of the \texttt{GroupChange} structure from
Figure~\ref{fig:group-change}, note the 64-bit integer
\texttt{0x8fbf37f586b0bc6e}. This is a randomly-generated number that serves as
a \textit{domain separator}. We refer to it below as unique type identifier
(UTID). Though domain specifiers are optional in the Snowpack language, when a
structure provides it, the snowpack compiler fills in five possible
cryptographic operations for the structure:
%
\begin{itemize}

  \item \texttt{PrefixedHash}(\textit{obj}): The object's UTID is big-endian encoded, then
  prepended to the object's binary Msgpack encoding. The hash of the combined message is returned.

  \item \texttt{Hmac}(\textit{obj}, \textit{key}): As above, a mesasge is formed out of the object's UTID
  concatenated with the encoding of the object itself. The combined message is the message input
  to the MAC function, and the key is passed through as the key.

  \item \texttt{SealIntoSecretBox}(\textit{obj}, \textit{nonce}, \textit{key}): The object's UTID is
  encoded and conctacted with the supplied \textit{nonce}. The new value is then used as the nonce
  passed into the encryption algorithm, along with the encoding of \textit{obj} and the supplied \textit{key}.

  \item \texttt{Sign}(\textit{obj}, \textit{key}): The object's UTID is prepended to an encoding
  of \textit{obj}; the combined message is then used as the message input, passed along to the 
  signature algorithm along with the supplied \textit{key}.

\end{itemize}
%
%
Public key encryption calls into \texttt{SealIntoSecretBox} with a random session key, so therefore uses the same
domain separation mechanism. Inverse operations for \texttt{Hmac}, \texttt{SealIntoSecretBox} and \texttt{Sign}
are also provided; they similarly supply UTIDs where necessary to ensure that verification and decryption
succeed.

The programmer must supply their own tooling to generate these UTIDs. Simple CLI tools or 
editor plugins suffice. However, FOKS provides two mechanisms to guararntee they remain unique
across the project. At compile-time, a simple tool examines all input files to guarantee that
no UTID contant appears twice. And at runtime, the compiler provides a list of all UTID constants
compiled from the protocol input files. The program fails an assertion if it sees any repeats.

\subsubsection{Variants}

We have seen Snowpack structures in Section~\ref{sec:snowpack-structures}. Another important
data type is the \textit{variant}, also known as a descriminated union. Figure~\ref{fig:link-inner}
shows an example from FOKS. The enumerated type \texttt{LinkType} has two possible values:
\texttt{GroupChange} for main chains, and \texttt{Generic} for subchains like team membership chains
and user settings chains. Based on the switch value, the enumerated type takes the form of 
a \texttt{GroupChange} or \texttt{GenericLink} object. Variants have many of the same restrictions
as structures: fields specified with `@i'-style slots should never be repurposed, though they 
can be dropped; also, new type possibilites can be added without breaking the protocol.

\begin{figure}[ht]
  \centering
  \begin{verbatim}
    variant LinkInner switch (t : LinkType) @0xacf9066572a9e7de {
      case GroupChange @0 : GroupChange;
      case Generic @1 : GenericLink;
    }\end{verbatim}
  \caption{A variant in the Snowpack language.}
  \label{fig:link-inner}
\end{figure}

\subsection{Cryptographic Primitives}
\label{sec:cryptography}

We have tried as much as possible to make boring, unopinionated cryptographic
decisions.

\subsubsection{Hashing and MAC'ing}

Throughout the system, hashing uses SHA512 truncated to 256
bits~\cite{rfc6234}. Message authentication codes are with HMAC~\cite{rfc2104}
over SHA512/256. HMAC is used for MAC'ing but also for commitments, and in 
general, any context where a pair of items are hashed together (one being the ``key''
and the other being the ``data'').

\subsubsection{Signatures and Post-Quantum Encryption}

For signing, we use EdDSA with the Ed25519 curve~\cite{eddsa}.  Public-key
encryption is a hybrid of Diffie-Hellman over
Curve25519~\cite{cryptoeprint:2011/368} and MLKEM~\cite{nist-fips-203} using a
construction similar to X-Wing~\cite{xwing}, but with a different binary
encoding format and constants. Thus, in practice, all device keys, PUKs,
PTKs, and so are are not a single keypair, but rather a triple: an EdDSA 
keypair, a Curve25519 keypair, and an MLKEM keypair. Wherever public keys
are introduced, an EdDSA signature over the Curve25519 and MLKEM public keys
is produced to bind them together.

The exact derivation of the hybrid encryption secret key is specified in 
Snowpack, using the structure shon in Figure~\ref{fig:xwing}. Hasn inputs are:
the domain separator (UTID); a version number; the shared key exchanged via KEM;
the shared Diffie-Hellman key; the receiver's public keys; and the sender's
public DH key. Though everywhere in the project we use SHA512/256, here we 
use SHA3 to follow the spirit of the X-wing specification.

\begin{figure}[ht]
  \centering
  \begin{verbatim}
struct HybridSecretKeySHA3Payload @0x8a9e327647262289 {
    version @0 : BoxHybridVersion;
    pqKemKey @1 : KemSharedKey;
    dhSharedKey @2 : DHSharedKey; 
    rcvr @3 : HEPK; // Hybrid Encryption Public Key = DH + KEM Public keys
    sndr @4 : DHPublicKey;
}
  \end{verbatim}
  \caption{Hybrid encryption secret key derivation in the Snowpack language.}
  \label{fig:xwing}
\end{figure}

\subsubsection{Key Derivation}
\label{sec:xwing-key-derivation}

As described just above, each public key in FOKS actually consists of three keypairs. However,
a single 32-byte secret seed suffices to generate all three, which simplifies secret key
management and backup keys.  The key derivation system again uses the Snowpack specification
system and simple HMAC-based key derivation. Figure~\ref{fig:key-derivation} shows the 
Snowpack structures and variants used. The derived key is the HMAC of the \texttt{KeyDerivation} object
with the secret 32-byte seed as the key.

\begin{figure}[ht]
  \centering
  \begin{verbatim}
enum KeyDerivationType {
    // Core types
    Signing @0;
    DH @1;
    SecretBoxKey @2;
    MLKEM @4;
    AppKey @5; // Used for different higher-level applications, like KV Store
}

enum AppKeyDerivationType {
    Enum @0;
    String @1;
}

enum AppKeyEnum {
    KVStore @0;
}

variant KeyDerivation switch (t: KeyDerivationType) @0xd35cdcc95caef674 {
    case MLKEM @4: Uint; // need 2 32-byte values to get a 64-byte seed
    default: void;
}
\end{verbatim}
  \caption{Structures and variants used in key derivation.}
  \label{fig:key-derivation}
\end{figure}

For example, to make a new PTK, the team adminstrator picks a random 32-byte
seed value.  When ever the PTK is used in a symmetric context (like for
encrypting older PTKs), the key derivation uses \texttt{KeyDerivationType =
SecretBoxKey} as an HMAC input.  Similarly for using the PTK as a signing key.
All derived keys are also 32-bytes with the exception of the ML-KEM key, which
needs 64-bytes. The two halfs of this derived key are generated with the same
mechanism, but using \texttt{MLKEM=0} and \texttt{MLKEM=1} in the
\texttt{KeyDervitation} object as the HMAC input.

\subsubsection{\Yubis}

FOKS supports hardware keys like YubiKey that support the Personal Identity
Verification~\cite{nist-sp-800-73-5} (PIV) standard.  This standard allows the
device to perform public key cryptopgrahic operations, like ECDSA and
Diffie-Hellman over the p256 elliptic curve~\cite{nist-fips-186-3}. Though we
have chosen the Ed25519 and Curve25519 curves for use everywhere, we now need to
accommodate another curve to fit the PIV standard. Too much ``agility'' has
proven problematic for other systems~\cite{jwt-none}, and we would like to avoid
it as much as possible with FOKS, but we make an exception here for a popular
hardware standard.

The bigger issue with YubiKeys is: what do to about post-quantum security?  To
date, we have not seen a wide release of an algorithm like ML-KEM to hardware
devices, and even if so, we'd like to support older, widely-deployed hardware. 

Since there are no perfoct solutions here, we have designed a PQ-secure system
around existing PIVs as follows:

\begin{enumerate}
  \item Extract a "secret" from the PIV module: pick an unused ``retired key management'' slot
  (0x82-0x95), and compute $g^{x^2}$ via the ECDH algorithm. Use this value as the seed
  to create a new ML-KEM keypair. Compute ML-KEM on user's computer after extracting the secret
  and deriving the keys.
  \label{step:pq-secret}

  \item Select a different retired key management slot to use for classical ECDH over curve p256.
  Compute ECDH as usually using the YubiKey's hardware.

  \item  Combine the secret keys from the previous two steps using the X-wing-style
  derivation scheme from Section~\ref{sec:xwing-key-derivation}.
\end{enumerate}

An important property of this system is that all of the relevant key material
lives on the YubiKey; none lives on the user's computer. The YubiKey is all the
user needs to recover important secrets, even if the computer is lost or suffers
data loss. Further, this system is no less secure than an encryption scheme
without PQ-security, as the classical ECDH computation still happens on the
YubiKey as normal. That is, we are not forcing the user to choose between PQ and
hardware security.  However, this scheme has an important shortcoming. If the
user later reuses the key management slot in Step~\ref{step:pq-secret} for a
different purpose, and exports the public key $g^x$ from the device, the scheme
is no longer PQ-secure.  A quantum computer could recover $x$ from $g^x$ and
then derive the ML-KEM secret key.  This shortcoming makes us long for better
hardware support for ML-KEM.  However, a mitigating factor here is that PIV is
an infrequently-used standard (and for instance is way less popular than FIDO2).
There are few competing applications using these features and key slots.

\subsubsection{High-Entropy Secret Phrase}
\label{sec:hesp}

For backup device keys (which FOKS users can write on pieces of paper), and 
exchanging provisioning secrets betwewen two computers, FOKS uses a simple
encoding scheme called the ``high-entropy secret phrase''. The pattern is 
a series of random words, each separated by a random number. All
words are chose from the BIP39 wordlist~\cite{bip39}. The exact
parameters depend on the application, and are shown in Table~\ref{tab:hesp-params}.

\begin{table}[ht]
  \centering
  \begin{tabular}{|c|c|c|c|c|c|}
    \hline Application & \# of Words & \# of Numbers & Number Range & Entropy \\
    \hline
    \hline
    Provisioning & 7 & 6 & $[0,2^8-1]$ & $7\cdot 11 + 6\cdot 8 = 125$ \\
    \hline
    Backup Device Key & 8 & 7 & $[0,2^{13}-1]$ & $8\cdot 11 + 7\cdot 13 = 179$ \\
    \hline
  \end{tabular}
  \caption{Parameters for high-entropy secret phrases.}
  \label{tab:hesp-params}
\end{table}


\subsection{Secret Key Management}

Secret keys derives from 32-byte seeds, which never leave the device they are created
on (with the possible exception of backup keys, which are written down on paper).
We discuss here how the FOKS client stores these secret seeds persistently. 

\subsubsection{Secure Enclaves}

Where possible, FOKS uses OS-specific secure enclaves. This is the simple case.
FOKS stores the actual 32-byte seeds in a FOKS-specific keyring file in the user's 
home directory. For each seed, FOKS picks a random 32-byte key to encrypt with, and,
if possible, stores that 32-byte key in the user's OS keyring. 

\subsubsection{Passphrase-based Management}

Though it's not encouragd, FOKS does offer a passphrased-based protection mechanism
for secret key seeds. As above, each secret seed gets its own secret-key wrapping 
material wrapping key (SKMWK). But instead of storing the SKMWKs in the OS keystore,
they are encrypted with a key derived from the user's passphrase. We have important 
design considerations for this system that make it quite complex:

\begin{enumerate}

\item If the user has two computers, $A$ and $B$, and the user changes his passphrase on A, when B comes online
 with the old passphrase, it has to decrypt with the new passphrase.

\item Keys encryped for old passphrases need to eventually be migrated to the new passphrase, so that if
an attacker gets the old passphrase and all server data, they still can't decrypt the key.  Of course
this is only possible if that computer $B$ comes back online after the change, but assuming that
the property should hold.

\item Passphrae recovery: to change the passphrase and recover keys, it is sufficient to know the latest PUK.
 Thus, having a backup paper key or a backup YubiKey should suffice to ``recover'' a passphrase
 and to allow the user to change it without knowing the old passphrase.

\item As with passphrases, if the PUK is updated, all machines with passphrase-encrypted keys
  should eventually rotate (when they come online) so that they cannot be decrypted with an old PUK
\end{enumerate}

\label{sec:passphrase}

\newcommand{\secretBox}[2]{\textsf{secretBox}(#1, #2)}
\newcommand{\dhbox}[3]{\textsf{dhBox}(#1, #2, #3)}
\newcommand{\pk}[1]{\textsf{publicKey}(#1)}
\newcommand{\sk}[1]{\textsf{secretKey}(#1)}

We describe the process through a small example: two rotations, one due to a PUK rotation,
and one due to a passphrase change. The net result is three different configurations
(the original, and the two following rotations). The general idea is that we have a new ``session''
key at every update, which is symmetrically encrypts the SKMWKs. The session key gets encrypted
twice: once for a key derived from the current passphrase, and one for the user's PUK. This
allows recovery of the SKMWKs with either the passphrase or the PUK:
%
  \begin{center}
  \begin{tabular}{|c|c|c|c|}
    \hline
     Key & Epoch 0 & Epoch 1 & Epoch 2 \\
     \hline
     \hline
      SKMWK & $r_0$ & $r_1$ & $r_2$ \\
      Session Key & $s_0$ & $s_1$ & $s_2$ \\
      Ephemeral DH Key & $t_0$ & $t_1$ & $t_2$ \\
      Passphrase & $p_0$ & $p_0$ & $p_1$ \\
      PUK & $u_0$ & $u_1$ & $u_1$ \\
     \hline
  \end{tabular}
\end{center}
%
At Epoch 0, we have the initial configuration, which consists of the following three encryptions:
%
\begin{align*}
e_0 &= \secretBox{r_0}{s_0} \\
f_0 &= \dhbox{s_0}{\pk{p_0}}{\sk{t_0}}, \pk{t_0} \\
g_0 &= \secretBox{[s_0,\pk{p_0}]}{u_0} \\
\end{align*}
%
$e_0$ is the encryption of the SKMWK $r_0$ for the session key $s_0$.  $f_0$ is the encryption 
of the session key for the user's current passphrase, $p_0$. To derive $\sk{p_0}$, we employ
a simple stretching algorithm and interpret the result as a Curve25519 secret key; then we
derive $\pk{p_0}$ from the secret key as usual. Finally, $g_0$ is the encryption of the session
key $s_0$ for the user's current PUK. We include $\pk{p_0}$ in the plaintext for reasons
we will see shortly.
%




\section{Applications}

The goal of the FOKS system is primarily: for a user or a group of users to
agree upon a sequence of cryptographic keys so they can perform authenticated,
end-to-end encryption of arbitrary data. As a secondary goal, the system
exposes a set of authorized signing keys to sign on behalf of the group,
so that changes can be properly attributed. From here, we can build
any number of applications. 

For instance, one can imagine an MLS system for group messaging~\cite{MLS} where
the chat keys are the cryptographic combination (via something like HMAC or
SHA3) of: (1) the root of the MLS ratchet tree; and (2) the most current PTKs
available for the FOKS group. In this way, one can simultaneously achieve
Signal-style forward secrecy and FOKS-style team and device management. 

For our first FOKS prototypes, we have focused instead of two important applications:
first, at the foundational level, a simple key-value store. Members of the team
can put and get key-value pairs to the FOKS server. Keys and values are encrypted
with authenticated encryption agains the team's PTKs. A second application, built
atop the first, is an end-to-end encrypted Git server, that is compatiable with
legacy Git clients. We describe them both below.

\subsection{The FOKS Key-Value Store}

\subsection{The FOKS Git Server}

\section{Related Work}

The initial inspiration for FOKS is the SUNDR project~\cite{sundr}, which first
originated the idea of a fork-consistent blockchain of edits facilitated by a
untrusted server.  Like Keybase~\cite{keybase}, FOKS applies this basic
architecture to the problem of key distribution, rather than the data those keys
might secure. Many other projects have riffed on this, like
CONIKS~\cite{melara2015coniks}, SEAMless~\cite{chase2019seemless}, 
ELEKTRA~\cite{cryptoeprint:2024/107}
and the widespread adoption of Key Transparency
Signal, WhatsApp and iMessage.  The question of federation has largely been
ignored, as these sytems all shared the basic architecture of a single upstream
server.

\section{Conclusion}

We have described, at a high level, the Federated Open Key Service (FOKS).
FOKS features multi-device support for users, arbitrary acyclic team graphs,
PQ-encryption, and federation. With these primtives, we can build applications
like end-to-end authenicated, encrypted key-value stores, and git hosting.
This service is currently operational~\cite{foks-app} and all 
source code is available on GitHub~\cite{foks-github}.
% Conclusion goes here

\bibliographystyle{plain}
\bibliography{refs}

\end{document}
