
\section{Applications}

The goal of the FOKS system is primarily: for a user or a group of users to
agree upon a sequence of cryptographic keys so they can perform authenticated,
end-to-end encryption of arbitrary data. As a secondary goal, the system
exposes a set of authorized signing keys to sign on behalf of the group,
so that changes can be properly attributed. From here, we can build
any number of applications. 

For instance, one can imagine an MLS system for group messaging~\cite{MLS} where
the chat keys are the cryptographic combination (via something like HMAC or
SHA3) of: (1) the root of the MLS ratchet tree; and (2) the most current PTKs
available for the FOKS group. In this way, one can simultaneously achieve
Signal-style forward secrecy and FOKS-style team and device management. 

For our first FOKS prototypes, we have focused instead of two important applications:
first, at the foundational level, a simple key-value store. Members of the team
can put and get key-value pairs to the FOKS server. Keys and values are encrypted
with authenticated encryption agains the team's PTKs. A second application, built
atop the first, is an end-to-end encrypted Git server, that is compatiable with
legacy Git clients. We describe them both below.

\subsection{The FOKS Key-Value Store}

\subsection{The FOKS Git Server}