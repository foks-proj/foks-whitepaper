
\section{Design}

FOKS is a classic client-server system. At a hight level, the clients
manage private keys, and the server manages public keys, encryptions of, 
shared secret keys, and encrypted data. Users generally trust their
servers to be online, available and not to intentionally sabotage
agreed-upon protocols. 

\subsection{System Architecture}

Much like HTTP or STMP, FOKS clients communicate with one or more servers,
depending on where users have accounts. They can safely ignore the other servers
in the system. Most communication is between client and server, and there is
little if any server-to-server or client-to-client communication.  This
property simplifies protocol upgrades and network configuration.

Each client can speak for many users, as users can have accounts on different
servers, or several accounts on the same server. By analogy, an email client can
server multiple emails accounts for the same user concurrently, say one for work
and one for personal use. Or a web browser might have different personae (with
different cookies, preeferences, passwords and history) for the same user.

Each of the users can of course have multiple devices, like a desktop, a laptop,
a phone, and a YubiKey. Additionally, users can have "backup devices", which can
be writted down on paper and stored in a safe place. The system recommends at
least two devices to prevent data loss. That is, these devices have private keys
that decrypt data, and the loss of the last key prevents decryption of the data.
Obviously there is a trade-off here: the more devices, the more likely the user
will lose one, or have one stolen; the fewer devices, the more likely the user
will lose all devices and therefore access to data. Some optimal middle ground
exists, but varies with the users and their behaviors. 

\subsection{Key Hierarchy}

The FOKS key hierarchy sits at the core of the system. It aims to provide users
with a sequence of symmetric keys shared across all of their devices, so that
they can store data encrypted with the latest key, and can decrypt (and
authenticate) data encrypted with older keys when necessary. Similarly, users in
a team should share secret keys that users outside their teams cannot see, allowing
them to share encrypted data via untrusted FOKS servers.

\subsubsection{Device Keys}

When a user sits down at a FOKS client to signup or provision a new device
for an existing account, she first creates a new key-pair specifically
for that device. The private key never leaves the device. She shares the public key
with the FOKS server, who eventually selectively shares it with user users.
We detail the exact cryptography in Section~\ref{sec:cryptography}.

Hardware keys that support the PIV protocol (like YubiKey version 5 and later)
can also be used as device keys. These devices get randomly-generated private
keys in the factory, written to one of 20 possible "slots." FOKS users select
a slot to use, and the client sends the corresponding public key to the FOKS 
server. Signing and decryption operations happen on the device against the chose
slot. 

\subsubsection{Per-User Keys (PUKS)}

Every user on the FOKS system has one of more per-user keys, or PUKS. A PUKS
is a randomly-generated key-pair whose private key is encrypted for each of
the device public keys. This way, all current devices can access the current
PUK secret key, and perform decryptions or signatures for the current PUK public
key. The client makes a new PUK every time the user revokes a device. The system
encrypts the old PUK secret keys for the new PUK secret key. This way,
a device that has access to the latest PUK can get access easily to all prior
PUKs.

Once the PUK sequence is estblished, the system has a convenient way to encrypt
a data for all of the user's device --- it simply encrypts the data for the 
user's latest PUK.

\subsubsection{Per-Team Keys (PTKs)}

Each team has a sequence of per-team-keys, or PTKs, which are analogous
to PUKs for users. Upon creation, a team gets a new random PTK. The 
client performing the creation sends the public part of the PTK to the server.
The private part of the PTK is encrypted for each member's latest PUK,
and therefore is available on each of the user's devices.

As with PUKs, data that the team shares is encrypted for the team's latest PTK,
and all members can decrypt it. As we will see in Section~\ref{sec:teams}, teams can
join other teams, but the key hierarchy works just the same. When team $A$ joins
team $B$, the secret part of team $B$'s PTK is encrypted for team $A$'s latest PTK,
so that all members of team $A$ can decrypt $B$'s PTK, and therefore, all of
$B$'s encrypted data.

\subsection{Key Roles}

\newcommand{\owner}{\textsf{owner}}
\newcommand{\admin}{\textsf{admin}}
\newcommand{\reader}{\textsf{reader}}
\newcommand{\role}[1]{\textsf{role}(#1)}


FOKS has a notion of a ``role'' for device keys, PUKs and PTKs. The roles
are: \owner, \admin, and \reader, but \reader{} keys have a "visibility level"
that varies between -32768 and 32767. There is a total ordering among key roles,
so that $\owner > \admin > \reader$, and between reader keys, $k_1 > k_2$ iff
$k_1$ has a higher visibility level than $k_2$.

The important property enforced is that we only encrypt PUK $k$ for device key
$j$ if $\role{k} \le \role{j}$, and similarly, we only encrypt PTK $k$ for PUK
$j$ if $\role{k} \le \role{j}$.

The idea here is that the owners of a group get to see all the keys; the admins
can see the admin and reader keys; and the readers can see keys at or below
their visiblility level. This configuration allows groups to have
lower-privileged members, and for users to have lower-privileged devices. At
Keybase, a similar but less-flexible property allows ``bots'' into teams, so that
all the members of the teams can interact with the bots, but the members had
channels to communicate that the bots aren't privvy to.  For now, all user
devices are at the \owner{} role, but we plan to relax this requirement in the
future.

\subsection{Data Structures}

We now have some basic motivation as to what the key system ought to achieve.
It ought to allow groups of devices, groups of users, or groups of users
and teams to share a secret encryption key. From there, they can share data
encrypted (and authenticated) with that key. But the question becomes,
how are users formulated from devices, and how are teams formulated from users
so that only desired members are in the group, especially if the server
behave maliciously?

For instance, a malicious server might fool a user into encrypting secret data
for an invalid device, or team administrator into encrypted data for an invalid
user.

\subsubsection{Signature Chains}

FOKS uses the same mechanism as Keybase here --- the signature chain (or sigchain for short).
The sigchain is a series of signed statements that form a cryptographic chain, meaning they
can only be replayed in the intended order. Replaying the chain allows a viewer to
confirm the chain appears how the author intended and wasn't tampered with, even if
the set of signers varies over time. Of course, signers do vary over time as
users add and remove devices, or as they add and remove members from teams.

Each user (and team) gets its own sigchain. The sigchain keeps an indellable record
of which keys can update the chain, and which PUKs or PTKs are currently
active for the user (or team).

\paragraph{Users} The first link in a sigchain is called the ``eldest'' link. For user sigchains,
the first device generates this link, generates the first PUK, and then computes
a signature over the following data:


\begin{enumerate}\itemsep0em
    \item \label{item:prev} The hash of the previous link in the chain (nil for the eldest)
    \item The current sequence number of the sigchain (which is 1 for the eldest link)
    \item A random commitment to the next tree location (see Section~\ref{sec:location-hiding})
    \item The user's ID and the server's host ID (see Section~\ref{sec:hostchains})
    \item \label{item:puk} The user's new PUK public key
    \item The user's new device key
    \item The current Merkle root hash (see Section~\ref{sec:merkle})
    \item A ``subchain tree location seed commitment`` (see Section~\ref{sec:subchains})
    \item A cryptographic commitment to the user's username (see Section~\ref{sec:commitments})
    \item A cryptographic commitment to the user's device name (picked by the user)
    \item The role of the new device (currently always \owner ).
    \item For Yubikeys, a public ``subkey' (see Section~\ref{sec:authentication})
\end{enumerate}

The public portion of the PUK computes this signature first; then the device key
computes a signature over the payload and the previous signature (signed with
the PUK). The client uploads the whole package as the user's eldest link.

Subsequent links proceed in largely the same way, with a few minor differences.
The previous hash (\ref{item:prev}) is the collision-resistent hash of the 
package uploaded in the previous step. New PUK public keys (\ref{item:puk}) do
not appear on device addition; and likewise, PUK signatures are only performed when
new PUKs are introduced, so would not be present on device addition.

\paragraph{Teams} 


\subsubsection{Hostchains}
\label{sec:hostchains}

\subsubsection{Merkle Tree}
\label{sec:merkle}

\subsubsection{Location Hiding}
\label{sec:location-hiding}

\subsubsection{Subchains}
\label{sec:subchains}


\subsection{Provisioning}

\subsubsection{Device-to-Device}

\subsubsection{Yubikey-to-Device}

\subsubsection{Device-to-Yubikey}

\subsection{Teams}
\label{sec:teams}

\subsubsection{Invitation Sequence}

\subsection{Key Rotations}

\subsubsection{Removal Keys}

\subsection{Cryptographic Primitives}
\label{sec:cryptography}

\subsubsection{Key Derivation}

\subsubsection{Yubikeys}


\subsubsection{PQ-KEM and PIV Support}

\subsection{Privacy}

\subsubsection{Blinding and Commitments}
\label{sec:commitments}

\subsubsection{Sigchain Visibility}

\subsection{Secret Key Management}

\subsubsection{Secure Enclaves}

\subsubsection{Passphrase-based Management}

\subsection{Beacon Server}