
\section{Introduction}

In recent years, Signal, iMessage and WhatsApp have proven out the customer
demand for end-to-end encrypted communication. But despite the success of these
systems, and their vast improvements over previous, less-secure systems,
important problems remain unresolved.  Most obviously, questions remain around
identity. In a recent, high-profile incident, United States government officials
misused Signal to mistakenly leak matters of national security to the
press~\cite{signal-hesgeth-leak}. Relying on phone numbers as identifiers is
only part of the problem; the larger issue, arguably, is that public identities
are hard to audit and map to public keys, and that large groups are even harder
to manage.

We see other issues: all the systems mentioned above and many others lock users
into a walled-garden with a centralized single provider. Outside observers lack
the ability to experiment with their own servers as part of a security audit.
Many of these systems lack an unconditional commitment to open-source
everywhere, closing off parts or all of their systems to third party scrutiny. 
Single-provider systems further suffer from vendor lock-in. As data is not
portable, and switching to a competing platform is not supported, the service
providers have the leverage to ``monetize'' their users as they see fit.

The authors have deep experience with the Keybase system~\cite{keybase}, which
came at these problems from a different angle. In Keybase, the initial focus was
on identity, multi-device support, and formation of auditable groups that could
evolve over time. But Keybase shows the same limitations as Signal, WhatsApp and
iMessage: it is stuck on a single-provider model.

This paper introduces a new system: the Federated Open Key Service (FOKS).  It
inherits much from these prior system but inhabits a different point in the
design space. FOKS provides secure key distribution for users who have multiple
devices. It allows those users to form groups, and unlike previous systems, for
those groups to join other groups. Whether managing a user's device cloud, or
managing the membership of a team, FOKS ensures that a malicious server cannot
inject invalid members, and that it cannot withhold important revocations and
deletions.  Many of these features are possible with previous systems, but
FOKS's major architectural advance is the support of federation. Anyone
can run a server in the FOKS network. Users can stay siloed on different
servers, or can form teams that span multiple servers. Federation gives users
more choice, control and better guarantees. Though servers cannot decrypt or
sign on behalf of users, they still can see metadata, and often are called upon
to protect user privacy. Therefore, companies or tight-knit groups have
good reason to run their own servers, and can do so in the system.

The aim here is to build a general protocol that can scale to the internet, 
owned by no single party. FOKS aims to be agnostic to hosting provider, much
like SMTP or HTTP. What those protocols have done for email and the web, FOKS
aims to do for cryptographic key management and distribution.

This paper introduces and describes the FOKS system. We cover a threat model
in Section~\ref{sec:threatmodel}, a system design in Section~\ref{sec:design}, 
the use of cryptography in Section~\ref{sec:crypto}, and some important applications
in Section~\ref{sec:apps}. The primary goal here is not academic novelty, but
rather to describe a system that embodies a unique
and quite useful set of trade-offs for end-to-end encrypted systems.
However, there are some, to our knowledge, new contributions:
%
(1) An exploration of key-rotation for teams that can form
      nearly arbitrary graphs across federated servers;
%
(2) a system for hiding identity and team updates in a larger
      transparency tree without the need
      for pseudo-random functions; and
%
(3) a new protocol specification language (called Snowpack) that enforces
  domain separation for cryptographic operations.
